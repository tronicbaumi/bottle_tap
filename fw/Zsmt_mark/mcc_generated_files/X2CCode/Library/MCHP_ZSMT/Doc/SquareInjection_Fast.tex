\XtoCBlock{SquareInjection\_Fast}
\label{block:SquareInjectionFast}
\missingfigure{SquareInjection\_Fast : SquareInjection\_Fast.png} 

\begin{XtoCtabular}{Inports}
Ix & d-current in estimated refercence frame x,y\tabularnewline
\hline
Iy & q-current in estimated refercence frame x,y\tabularnewline
\hline
Sync & synchronization with slowTaskISR;

First detected rising egde of the Sync-input activates the voltage injection\tabularnewline
\hline
EnableInj & Enables and disables injection of HF-voltage. 

Usually used in combination with sensorless bEmf-algortihms @higher speeds. For applications solely controlled by hf-injection, set this input to constant 1.



EnableInj = 0: hf-voltage injection is disabled. Ud and angle-error dPhi are set to zero.

EnableInj != 0:  hf-voltage is enabled\tabularnewline
\hline
Enable & Enable = 0: Outputs are zero.

Enable != 0: Initial rotor angle is determined, PLL and controller enabled\tabularnewline
\hline
\end{XtoCtabular}


\begin{XtoCtabular}{Outports}
Ud & This Output has 2 operational meanings:

@startUp: ud is the amplitude of the voltage-phasor for the calculation of the initial rotor angle.

@Injection: once Injection is activated, ud is the amplitude of the square-wave voltage injected. \tabularnewline
\hline
PhiInit & Initial rotor angle\tabularnewline
\hline
EnablePLL & Enable signal for PLL\tabularnewline
\hline
EnableCtrl & Enable signal for motor-controllers\tabularnewline
\hline
dPhi & Angle-error of the estimated reference frame x,y. k\epsilon \sin (2\Delta\varphi)\tabularnewline
\hline
\end{XtoCtabular}

\begin{XtoCtabular}{Mask Parameters}
VoltPhasor & Angle Detection: Amplitdue of voltage pulses used for determination of initial rotor anlge\tabularnewline
\hline
T1 & Angle Detection: Time after enable of first voltage-pulse\tabularnewline
\hline
deltaT & Angle Detection: Time inbetween voltage pulses\tabularnewline
\hline
Tpulse & Angle Detection: Time of voltage-pulse to be at the given amplitude\tabularnewline
\hline
DelayPLL & Time delay for enabling PLL\tabularnewline
\hline
DelayCtrl & Time Delay for enabling PLL [s]\tabularnewline
\hline
uInj & voltage amplitude of injected square wave\tabularnewline
\hline
uDCmax & Normalization of Udc link voltage\tabularnewline
\hline
ts\_fact & \tabularnewline
\hline
\end{XtoCtabular}

\subsubsection*{Description:}
* AngleDetection: calculates rotor-angle at init-time;

* Enable Sequence: realizes an adjustable Delay-Sequence for enabling PLL and controller;

* Voltage Injection: injects a square-wave voltage signal with fPWM/2 and adjustable amplitude;



BE AWARE:

This Block needs the SquareInjection\_Slow-Block to work properly!

% include optional documentation file
\InputIfFileExists{SquareInjection_Fast_Info.tex}{\vspace{1ex}}{}

\subsubsection*{Implementations:}
\begin{tabular}{l l}
\textbf{FiP16} & \tabularnewline
\end{tabular}

\XtoCImplementation{FiP16}
\nopagebreak[0]



\begin{XtoCtabular}{Inports Data Type}
Ix & int16\tabularnewline
\hline
Iy & int16\tabularnewline
\hline
Sync & int16\tabularnewline
\hline
EnableInj & bool\tabularnewline
\hline
Enable & bool\tabularnewline
\hline
\end{XtoCtabular}

\begin{XtoCtabular}{Outports Data Type}
Ud & int16\tabularnewline
\hline
PhiInit & int16\tabularnewline
\hline
EnablePLL & bool\tabularnewline
\hline
EnableCtrl & bool\tabularnewline
\hline
dPhi & int16\tabularnewline
\hline
\end{XtoCtabular}

\ifdefined \AddTestReports
\InputIfFileExists{\XcHomePath/Library/MCHP_LSHT/Doc/Test_SquareInjection_Fast_FiP16.tex}{}{}
\fi
